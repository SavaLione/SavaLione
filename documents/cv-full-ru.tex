% SPDX-License-Identifier: CC-BY-4.0
%
% Copyright (C) 2022-2025 Savelii Pototskii (savalione.com)
%
% Derived from SFIUCR template CV/resume:
%
% Copyright (C) 2022 ComplexityExplorer (https://www.overleaf.com/latex/templates/sfiucr-template-cv-slash-resume/xqvshnpvtsbv)

% Description: It's a full resume
% Updated: 2025-05-08

\documentclass[a4paper,11pt]{article}

% You can use this tools to find font in your system:
%   Information about the font: otfinfo -i /path/to/font
%   Installed fonts: fc-list
%   Update fonts: fc-cache -f -v
%   On Ubuntu you can store fonts here: /usr/local/share/fonts
\usepackage{fontspec}
\setmainfont{Roboto}[
    UprightFont={Roboto-Regular},
    BoldFont={Roboto-Bold},
    ItalicFont={Roboto-Italic}
]

\usepackage{latexsym}
\usepackage[empty]{fullpage}
\usepackage{titlesec}
\usepackage{marvosym}
\usepackage[usenames,dvipsnames]{color}
\usepackage{verbatim}
\usepackage{enumitem}
\usepackage[hidelinks]{hyperref}
\usepackage{fancyhdr}
\usepackage[english]{babel}
\usepackage{tabularx}
\usepackage{multicol}

% Document properties (pdf)
\hypersetup{
%    bookmarks=false,                            % show bookmarks bar?
    unicode=true,                               % non-Latin characters in Acrobat’s bookmarks
    pdftoolbar=true,                            % show Acrobat’s toolbar?
    pdfmenubar=true,                            % show Acrobat’s menu?
    pdffitwindow=false,                         % window fit to page when opened
%    pdfstartview={FitW},                       % fits the width of the page to the window
    pdftitle={Савелий Потоцкий - Подробное резюме}, % title
    pdfauthor={Савелий Потоцкий},              % author
    pdfsubject={Резюме - подробное},                 % subject of the document
%    pdfcreator={\LaTeX},                       % creator of the document
    pdfproducer={\LaTeX},                       % producer of the document
    pdfkeywords={Резюме,} {resume, } {cv, } {application}, % list of keywords
    pdfnewwindow=true,                          % links in new window
    colorlinks=false,                           % false: boxed links; true: colored links
    linkcolor=red,                              % color of internal links
    citecolor=green,                            % color of links to bibliography
    filecolor=magenta,                          % color of file links
    urlcolor=cyan                               % color of external links
}


\pagestyle{fancy}
\fancyhf{}
\fancyfoot{}
\setlength{\footskip}{10pt}
\renewcommand{\headrulewidth}{0pt}
\renewcommand{\footrulewidth}{0pt}

\addtolength{\oddsidemargin}{0.0in}
\addtolength{\evensidemargin}{0.0in}
\addtolength{\textwidth}{0.0in}
\addtolength{\topmargin}{0.2in}
\addtolength{\textheight}{0.0in}

\urlstyle{same}

%\raggedbottom
\raggedright
\setlength{\tabcolsep}{0in}

\titleformat{\section}{
    \it\vspace{3pt}
}{}{0em}{}[\color{black}\titlerule\vspace{-5pt}]

\newcommand{\resumeItem}[1]{
    \item{
                {#1 \vspace{-4pt}}
          }
}

\newcommand{\resumeSubheading}[4]{
    \vspace{-2pt}\item
    \begin{tabular*}{0.97\textwidth}[t]{l@{\extracolsep{\fill}}r}
        \textbf{#1} & #2 \\
        \textit{\small #3} & \textit{\small #4} \\
    \end{tabular*}\vspace{-10pt}
}

\newcommand{\resumeSubItem}[1]{\resumeItem{#1}\vspace{-3pt}}
\renewcommand\labelitemii{$\vcenter{\hbox{\tiny$\bullet$}}$}
\newcommand{\resumeSubHeadingListStart}{\begin{itemize}[leftmargin=0.15in, label={}]}
        \newcommand{\resumeSubHeadingListEnd}{\end{itemize}}
\newcommand{\resumeItemListStart}{\begin{itemize}}
        \newcommand{\resumeItemListEnd}{\end{itemize}\vspace{-2pt}}

\begin{document}

\begin{center}
    {\LARGE Савелий Потоцкий} \\ \vspace{0pt}
    \begin{multicols}{2}
        \begin{flushleft}
            % \large{} \\
            % \large{} \\
        \end{flushleft}

        \begin{flushright}
            \href{https://savalione.com}{savalione.com} \\
            %\large{savalione.com} \\

            \href{mailto:{savelii.pototskii@gmail.com}} \large{savelii.pototskii@gmail.com}
        \end{flushright}
    \end{multicols}
\end{center}

%-----------EDUCATION-----------
\section{Образование}
\resumeSubHeadingListStart

\resumeSubheading
    {Уральский Федеральный Университет}{Сент. 2023}
    {PhD в области информационных технологий и телекоммуникаций}{Екатеринбург, Россия}
\resumeSubheading
    {Уральский Федеральный Университет}{Сент. 2021 -- Июнь 2023}
    {Магистр прикладной информатики}{Екатеринбург, Россия}
\resumeSubheading
    {Уральский Федеральный Университет}{Сент. 2017 -- Июнь 2021}
    {Бакалавр в области разработки программного обеспечения}{Екатеринбург, Россия}

\resumeSubHeadingListEnd

%-----------RESEARCH EXPERIENCE-----------
% \section{Research Experience}
% \resumeSubHeadingListStart

% \resumeSubheading
%         {Research experience title}{Month, Year started -- Month, Year ended or "Present" if ongoing}
%         {University, Company or Organization}{City, State, optional Country}
%       \resumeItemListStart
%         \small\resumeItem{very short description of one thing you did}
%         \resumeItem{very short description of another thing you did}
%         \resumeItem{very short description if there was another thing you did}
%     \resumeItemListEnd

% \resumeSubHeadingListEnd

%-----------OTHER EXPERIENCE-----------
\section{Опыт работы}
\resumeSubHeadingListStart

\resumeSubheading
    {Ведущий инженер-программист}{Июль 2022}
    {ООО <<Древ Мастер>> (на постоянной основе)}{Екатеринбург, Россия}
    \resumeItemListStart
        \small\resumeItem{Проектирование и разработка серверного программного обеспечения.}
        \resumeItem{Разработка архитектуры клиент-серверных приложений.}
        \resumeItem{Разработка сетевой архитектуры.}
        \resumeItem{Оптимизация программного обеспечения на C++ и модульное тестирование.}
    \resumeItemListEnd

\resumeSubheading
    {Программист}{Февр. 2024}
    {ИММ УрО РАН (неполная занятость)}{Екатеринбург, Россия}
    \resumeItemListStart
        \small\resumeItem{Разработка программного обеспечения.}
        \resumeItem{Разработка встроенных систем Linux.}
        \resumeItem{Разработка сетевого программного обеспечения.}
    \resumeItemListEnd

\resumeSubheading
    {Учебный мастер}{Март 2023}
    {Уральский Федеральный Университет (неполная занятость)}{Екатеринбург, Россия}
    \resumeItemListStart
        \small\resumeItem{Проектирование и разработка программного обеспечения.}
        \resumeItem{Разработка веб-приложений.}
        \resumeItem{Разработка архитектуры клиент-серверных приложений.}
        \resumeItem{Руководство студенческими проектами.}
        \resumeItem{Преподавание.}
    \resumeItemListEnd

\resumeSubheading
    {Инженер-программист}{Март 2018 -- Июль 2022}
    {ООО <<Древ Мастер>> (на постоянной основе)}{Екатеринбург, Россия}
    \resumeItemListStart
        \small\resumeItem{Проектирование и разработка серверного программного обеспечения.}
        \resumeItem{Системное администрирование комплекса серверов.}
        \resumeItem{Разработка модулей для сайтов, баз данных и веб-серверов.}
    \resumeItemListEnd

%\resumeSubheading
%    {Network engineer}{Apr, 2015 started -- Feb, 2018 ended}
%    {Freelance}{Екатеринбург, Россия}
%    \resumeItemListStart
%        \small\resumeItem{server administration (linux, bsd)}
%        \resumeItem{network software development}
%    \resumeItemListEnd

\resumeSubHeadingListEnd

%-----------RESEARCH PRESENTATIONS-----------
% \section{Research Presentations} 
% \begin{itemize}[leftmargin=0.15in, label={}]
%     \normalsize{\item{
%     {citation in your preferred format -- include all authors who contributed}{} \\
%     {citation in your preferred format -- include all authors who contributed}{} 
% }}
%  \end{itemize}

%-----------AWARDS & HONORS-----------
% \section{Awards \& Honors} 
% \resumeSubHeadingListStart
%     \resumeSubheading
%     {Title or brief description of the award}{}
%     {University, Sponsor or Organization}{year(s)}
%     \resumeSubheading
%     {Title or brief description of the award}{}
%     {University, Sponsor or Organization}{year(s)}
%     \resumeSubheading
%     {Title or brief description of the award}{}
%     {University, Sponsor or Organization}{year(s)}
% \resumeSubHeadingListEnd

%-----------SOFTWARE DEVELOPMENT SKILLS-----------
\section{Навыки разработки программного обеспечения}
\begin{itemize}[leftmargin=0.15in, label={}]
    \normalsize{\item{
                    \textbf{Основной язык программирования}{: C++} \\
                    \textbf{Дополнительные языки программирования}{: C, PHP, Python} \\
                    \textbf{Компиляторы}{: GCC, Clang, MinGW} \\
                    \textbf{Библиотеки}{: STL, Boost, Protobuf, spdlog, SDL2, ImGui, libcurl, SQLite} \\
                    \textbf{Инструменты}{: CMake, Git, Bash, Google Test, Doxygen, Google Benchmark} \\
                    \textbf{Ядра}{: Linux, BSD} \\
                    \textbf{HPC}{: OpenCL, OpenMP, Khronos SYCL} \\
                    \textbf{Архитектуры}{: X86-64, RISC-V, ARM, SPARC V9} \\
              }}
\end{itemize}

%-----------DEVOPS SKILLS-----------
\section{Навыки DevOps}
\begin{itemize}[leftmargin=0.15in, label={}]
    \normalsize{\item{
                    \textbf{Веб-серверы}{: Nginx, Lighttpd} \\
                    \textbf{Веб-инструменты}{: Grafana, Asterisk, Jekyll} \\
                    \textbf{CI/CD}{: Jenkins} \\
                    \textbf{Мониторинг}{: Zabbix} \\
                    \textbf{Контейнеры и VM}{: Linux LXC, Incus, Docker, FreeBSD jail, KVM, Proxmox} \\
                    \textbf{Базы данных}{: SQLite, MariaDB (MySQL), PostgreSQL} \\
                    \textbf{Инструменты}{: OpenZFS, Systemd, VPP} \\
                    \textbf{ОС и дистрибутивы}{: Ubuntu LTS, Debian, FreeBSD, NetBSD, pfSense} \\
                    \textbf{Серверы}{: Hewlett-Packard, Dell, Sun Microsystems} \\
              }}
\end{itemize}

%-----------Hardware Engineering Skills -----------
\section{Навыки разработки аппаратного обеспечения}
\begin{itemize}[leftmargin=0.15in, label={}]
    \normalsize{\item{
                    \textbf{Инструменты}{: KiCad, OpenOCD, логические анализаторы} \\
                    \textbf{RTOS}{: FreeRTOS, NuttX} \\
                    \textbf{Микроконтроллеры}{: Raspberry Pi Pico, STM32, ESP32} \\
                    \textbf{Одноплатные компьютеры}{: Raspberry Pi Zero 2w, Milk-v Duo, Orange Pi} \\
                    \textbf{FPGA}{: Tang Nano 9K (Gowin GW1NR-9), Tang Nano 1K (Gowin GW1NR-1)} \\
                    \textbf{Архитектуры}{: X86-64, RISC-V, ARM, SPARC V9} \\
              }}
\end{itemize}

%-----------OTHER SKILLS-----------
\section{Другие навыки}
\begin{itemize}[leftmargin=0.15in, label={}]
    \normalsize{\item{
                    \textbf{Языки}{:}
                    \begin{itemize}
                        \item Английский -- C1.
                        \item Русский -- родной.
                    \end{itemize}
                    \textbf{Системы разметки текста}{: \LaTeX} \\
              }}
\end{itemize}

% Define a new command \talkentry
% Takes two arguments:
% #1: Text for the left column (talk title)
% #2: Text for the right column (institution and date)
\newcommand{\talkentry}[2]{%
    \item % This command will generate a new list item
    \noindent % Ensures the tabularx starts at the beginning of the item line
    \begin{tabularx}{\linewidth}[t]{@{} X @{\hspace{2em}} r @{}}
        #1 & % Argument 1: Talk title
        #2   % Argument 2: Institution (bolded) and date
    \end{tabularx}%
}

%-----------TALKS-----------
\section{Выступления}
\begin{enumerate}[leftmargin=0.35in, labelindent=0pt, itemsep=1.5pt]
\normalsize{
    \talkentry{Инфраструктура для управления многосевыми роботами-манипуляторами в среде ROS}{\textbf{ИММ УрО РАН} -- 2021-05-20}
    \talkentry{Разработка маршрутизатора передачи данных на основе векторов для увеличения пропускной способности сети}{\textbf{ИММ УрО РАН} -- 2022-10-20}
    \talkentry{Модель маршрутизаторов на основе векторов}{\textbf{ИММ УрО РАН} -- 2023-04-27}
    \talkentry{Микроконтроллер Milk-v Duo на архитектуре RISC-V}{\textbf{ИММ УрО РАН} -- 2024-02-01}
    \talkentry{Отчёт по работе с платой для разработки Yahboom K210 Visual Recognition Module, результаты работы с ROS (Robot Operating System), анализ Apache NuttX, введение в ld scripts (скрипты линковщика), SSG Jekyll, работа с физическими серверами производства компании Hewlett-Packard}{\textbf{ИММ УрО РАН} -- 2024-05-23}
    \talkentry{История популярных микропроцессоров, микроконтроллеров, инструменты для работы с ними и операционные системы реального времени}{\textbf{ИММ УрО РАН} -- 2024-10-24}
    \talkentry{Разработка и анализ масштабируемых архитектур для обработки данных на основе малопотребляющих FPGA Gowin GW1NR-9}{\textbf{УрФУ ИМКН} -- 2024-12-19}
    \talkentry{Разработка электронного устройства}{\textbf{УрФУ ИМКН} -- 2024-02-13}
    \talkentry{Разработка электронного устройства}{\textbf{ИММ УрО РАН} -- 2025-04-20}
}
\end{enumerate}

%-----------PUBLICATIONS-----------
\section{Публикации}
\begin{enumerate}[leftmargin=0.35in, labelindent=0pt, itemsep=0pt]
\normalsize{
    \talkentry{Всемирная паутина как информационно-коммуникационная среда функционирования интернет-аукционов по продаже антиквариата}{2021-04-12}
    \talkentry{Разработка управляющего ПО и визуального эмулятора для многоосевого промышленного робота \textit{(диссертация бакалавра)}}{2021-06-24}
    \talkentry{Сравнение производительности скалярного программного маршрутизатора с программным маршрутизатором на основе векторов}{2023-05-28}
    \talkentry{Сравнение генераторов сетевого трафика в целях оптимизации процесса тестирования телекоммуникационного оборудования}{2023-05-28}
    \talkentry{Разработка маршрутизатора передачи данных на основе векторов для увеличения пропускной способности сети \textit{(диссертация магистра)}}{2023-07-01}
}
\end{enumerate}

%-----------TEACHING-----------
\section{Teaching}
\resumeSubHeadingListStart
    \resumeSubheading
        {Уральский Федеральный Университет}{Март 2023 -- Май 2023}
        {Институт фундаментального образования}{}
        \resumeItemListStart
            \small\resumeItem{Руководил студенческими командами в разработке масштабируемых веб-приложений.}
        \resumeItemListEnd

    \resumeSubheading
        {Уральский Федеральный Университет }{Сент. 2023 -- Дек. 2023}
        {Институт фундаментального образования}{}
        \resumeItemListStart
            \small\resumeItem{Направлял студенческие проекты, сосредоточенные на создании веб-приложений для онлайн-образовательных платформ.}
            \resumeItem{Консультировал студентов в проектах по разработке игр, предлагая экспертные знания в области структуры кода, игрового дизайна, выбора библиотек и подходов к разработке.}
        \resumeItemListEnd

    \resumeSubheading
        {Уральский Федеральный Университет}{Февр. 2024 -- Май 2024}
        {Институт фундаментального образования}{}
        \resumeItemListStart
            \small\resumeItem{Преподавал основы программирования на Java, делая акцент на его практическом применении в веб-разработке.}
            \resumeItem{Поддерживал разработку студенческих игр, оказывая специализированную помощь в игровой механике и эффективном применении программных библиотек.}
            \resumeItem{Осуществлял надзор за созданием и интеграцией веб-образовательных платформ студенческими командами в инфраструктуру университета.}
        \resumeItemListEnd
\resumeSubHeadingListEnd

%-----------OTHER INTERESTS-----------
\section{Другие интересы}
\begin{itemize}[leftmargin=0.15in, label={}]
    \normalsize{\item{
            \textbf{3D-печать}{: Опыт работы с различными 3D-принтерами благодаря доступу в университете и личным проектам} \\
            \textbf{Спорт}{: Бег и велоспорт} \\
            \textbf{Открытое ПО}{: Я глубоко заинтересован в представлении преимуществ программного обеспечения с открытым исходным кодом} \\
            \textbf{Фотография}{: Навыки использования зеркальных фотоаппаратов, применяю это для визуальной документации личных проектов и исследования творческих возможностей} \\
    }}    
\end{itemize}

%-----------LINKS-----------
\section{Ссылки}
\begin{itemize}[leftmargin=0.15in, label={}]
    \normalsize{\item{
                    \textbf{Email}{:  savelii.pototskii@gmail.com} \\
                    \textbf{LinkedIn}{:  \href{https://linkedin.com/in/savalione/}{linkedin.com/in/savalione/}} \\
                    \textbf{Github}{:  \href{https://github.com/SavaLione}{github.com/SavaLione}} \\
                    \textbf{Личная веб-страница}{: \href{https://savalione.com}{savalione.com} } \\
              }}
\end{itemize}

%-----------ADDITIONAL LINKS-----------
\section{Дополнительные ссылки}
\begin{itemize}[leftmargin=0.15in, label={}]
    \normalsize{\item{
                    \textbf{ORCID}{: \href{https://orcid.org/0009-0006-5105-465X}{orcid.org/0009-0006-5105-465X} } \\
                    \textbf{Google for developers}{: \href{https://g.dev/savalione}{g.dev/savalione} } \\
                    \textbf{Linux foundation}{: \href{https://openprofile.dev/profile/SavaLione}{openprofile.dev/profile/SavaLione} } \\
                    \textbf{Credly}{: \href{https://www.credly.com/users/sava}{@sava} } \\
              }}
\end{itemize}

\end{document}
